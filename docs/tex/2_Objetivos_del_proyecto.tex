\capitulo{2}{Objetivos del proyecto}

Este apartado explica de forma precisa y concisa cuales son los objetivos que se persiguen con la realización del proyecto. Se puede distinguir entre los objetivos marcados por los requisitos del software a construir y los objetivos de carácter técnico que plantea a la hora de llevar a la práctica el proyecto.

El objetivo principal que persigue este proyecto es el desarrollo de un entorno domótico que permita la visualización de cámaras en el hogar a través de una aplicación Android, todo ello a través de la comunicación entre un servidor que realice la gestión y el propio dispositivo Android.

El servidor debe establecer la conexión con las cámaras y ser capaz de retransmitir la imagen al dispositivo Android, conectado a la misma red WiFi. 

\section{Objetivos generales}\label{objetivos-generales}

\begin{itemize}
\tightlist
\item
  Desarrollar una aplicación para el entorno Android que permita la monitorización de cámaras dispuestas en un hogar.
\item
  Desarrollar y desplegar un servidor que gestione el entorno dómito (la conexión con las cámaras y con la aplicación Android).
\item
  Facilitar al usuario la monitorización mediante una interfaz fácil y sencilla.
\item
  Reutilizar dispositivos en desuso.
\item
  Utilizar dispositivos de bajo consumo.
\end{itemize}

\section{Objetivos técnicos}\label{objetivos-tecnicos}

\begin{itemize}
\tightlist
\item
  Desarrollar un servidor con el lenguaje Python que gestione toda la complejidad.
\item
  Desarrollar una aplicación en el entorno Android con soporte para API 16 y superiores.
\item
  Emplear la herramienta Gradle para la automatización del proceso de construcción y compilación de software de la aplicación Android.
\item
  Utilizar Git como sistema de control de versiones distribuido junto con
la plataforma GitHub.
\item
  Aprovechar las posibilidades que ofrecen las herramientas de integración continua como SonarQube en el repositorio.
\end{itemize}

\section{Objetivos personales}\label{objetivos-personales}

\begin{itemize}
\tightlist
\item
  Mostrar la capacidad de los conocimientos adquiridos durante el periodo universitario.
\item
  Aprender a manejar metodologías y herramientas innovadoras que son utilizadas en el mercado laboral.
\item
  Mejorar mi habilidad en el desarrollo de aplicaciones en el entorno Android.
\item
  Profundizar en los conceptos aprendidos sobre Python en mi formación universitaria.
\end{itemize}

