\capitulo{4}{Técnicas y herramientas}

\section{Herramientas de control de versiones}

\subsection{Github}

GitHub es una plataforma que nos permite gestionar y organizar nuestros proyectos y está basada en la nube, incorporando las funciones de control de versiones que proporciona Git. Esta herramienta posibilita a los desarrolladores poder almacenar y administrar su código, realizar un registro y control de los cambios sobre el código almacenado. Es una de las herramientas más populares entre los desarrolladores, cuenta con más de 100 millones de repositorios, y la mayoría de ellos son de código abierto.\\
He utilizado GitHub para el alojamiento de mi proyecto en el repositorio ``https://github.com/fmv1001/LocalStream''.

\subsubsection{Git Bash}

Git Bash es una aplicación para Windows para la emulación de la línea de comandos de Git a través de una shell. Una shell es un intérprete de comandos que provee una interfaz de usuario para la comunicación con el sistema operativo. \\
Esta aplicación me ha servido para subir cambios al respositorio de GitHub a través de los comandos git.



\section{Herramientas de gestión de proyectos}

\subsection{ZenHub}

ZenHub es una plataforma para la gestión ágil de proyectos que se integra con github, funcionando como aplicación nativa en su interfaz. Te ayuda a planificar tu proyecto dentro de GitHub, automatiza el flujo de trabajo. Más del 75\% de los desarrolladores que usan ZenHub en sus proyectos dicen que ZenHub mejora su enfoque y ayuda en el envío de un mejor software en un menor periodo de tiempo, y el 65\% informan de proyectos con un mejor alcance.\\
ZenHub me ha permitido gestionar mi proyecto, ayudándome con la planificación del mismo, para cumplir los tiempos de entrega del mismo.



\section{Metodologías}

\subsection{Modelo Cliente-Servidor}

Modelo Cliente-Servidor.



\section{Herramientas de evaluación de código}

\subsection{SonarQube}

SonarQube es una plataforma de código abierto desarrollada en Java que nos permite realizar análisis de código con diferentes herramientas de forma automatizada. Usa diferentes herramientas de análisis estático de código fuente como pueden ser Checkstyle, PMD o FindBugs para obtener métricas que pueden ayudar a mejorar la calidad del código de un programa.\\
En este proyecto se ha usado la herramienta Sonar-Scanner de SonarQube, para la evaluación del código tanto de la aplicación Android en java como del servidor desarrollado en python.



\section{Herramientas de documentación}

\subsection{\LaTeX}

\LaTeX es un sistema de software libre de composición tipográfica de alta calidad, orientado a la producción de documentación técnica y científica. \LaTeX es el estándar de facto para la comunicación y publicación de documentos científicos gracias a sus características, posibilidades y calidad profesional.\\
He usado \LaTeX para el desarrollo de tanto este documento como de los anexos.



\section{Lenguajes de programación}

\subsection{Python}

Python es un lenguaje de programación ampliamente utilizado por empresas de casi todo el mundo para la construcción de aplicaciones web, analización de datos, automatización de operaciones y creación de aplicaciones empresariales con alta fiabilidad y escalabilidad.\\
Para desarrollar el servidor he utilizado la versión 3.0 del lenguaje Python.

\subsection{Java}

Java es un lenguaje de programación orientado a objetos y basado en clases que se usa para el desarrollo de aplicaciones, diseñado para tener la menor cantidad de dependencias de implementación posibles. Además es uno de los lenguajes de programación utilizados para el desarrollo de aplicaciones para el sistema operativo Android. El código compilado de Java puede ejecutarse en cualquier plataforma que tenga instalada la máquina virtual java, sin la necesidad de volver a ser compilado.\\
Java es el lenguaje que he escogido para la construcción de la aplicación Android debido a mi familiaridad con este lenguaje.

\subsection{SQL}

SQL (Structured Query Language) es un lenguaje declarativo que ha sido diseñado para la administración y recuperación de información de sistemas de gestión de bases de datos relacionales. Este lenguaje es conservado por el organismo ANSI (American National Standards Institute).\\
El lenguaje SQL ha sido muy útil en este proyecto a la hora de respaldar al información necesaria tanto en el servidor como en la aplicación.



\section{Entornos de desarrollo integrado (IDE)}

\subsection{Android Studio}

Android Studio es el entorno de desarrollo integrado (IDE) oficial para el desarrollo de aplicaciones para el sistema operativo Android y está basado en IntelliJ IDEA (IDE de Java desarrollado por JetBrains). Además del potente editor de códigos y las herramientas para desarrolladores de IntelliJ, Android Studio ofrece incluso más funciones que aumentan tu productividad cuando desarrollas apps para Android, entre otras tenemos: un sistema de compilación flexible basado en Gradle o la integración con GitHub y plantillas de código para ayudarte a compilar funciones de apps comunes y también importar código de muestra.\\
Android Studio es el IDE oficial para desarrollar aplicaciones en Andorid y por eso lo he escogido para mi proyecto.

\subsection{Visual Studio Code}

Visual Studio Code es un editor de código fuente desarrollado por Microsoft para Windows, Linux y macOS. Incluye soporte para la depuración, control integrado de Git, resaltado de sintaxis, finalización inteligente de código, fragmentos y refactorización de código. Es gratuito y de código abierto aunque la descarga oficial está bajo software privativo e incluye características personalizadas por Microsoft.\\ 
Para el despliegue del servidor he utilizado esta aplicación open source de Microsoft.




\section{Herramientas de automatización de compilación del código}

\subsection{Gradle}

Gradle, es una herramienta que permite la automatización de compilación de código abierto, la cual se encuentra centrada en la flexibilidad y el rendimiento. Los scripts de compilación de Gradle se escriben utilizando Groovy o Kotlin DSL (Domain Specific Language). Gradle tiene una gran flexibilidad y nos deja hacer usos otros lenguajes y no solo de Java, también cuenta con un sistema de gestión de dependencias muy estable. Además es el sistema de compilación oficial para Android y cuenta con soporte para diversas tecnologías y lenguajes.



\section{Protocolos de comunicación}

\subsection{UDP}

El protocolo de datagramas de usuario (User Datagram Protocol o UDP) es un protocolo de la capa de transporte basado en el intercambio de datagramas. Permite el envío de datagramas a través de la red sin que se haya establecido previamente una conexión, ya que el propio datagrama incorpora suficiente información de direccionamiento en su cabecera.\\
Gracias a este protocolo el servidor envía las imágenes al dispositivo móvil Android sin que este tengo que confirmar que lo ha recibido.

\subsection{TCP / IP}

TCP es un protocolo de capa de transporte utilizado por aplicaciones que requieren una entrega garantizada. Se trata de un protocolo de ventana deslizante que permite gestionar tanto los tiempos de espera como las retransmisiones. TCP establece una conexión virtual full duplex entre dos puntos finales. Cada punto final está definido por una dirección IP y un número de puerto TCP. El funcionamiento de TCP se implementa como una máquina de estado finito.\\
Para que el servidor y la aplicación se comuniquen a la hora de realizar alguna tarea, este protocolo ha permitido que en ningún momento se pierda la conexión y en ese caso los dos serán conscientes y actuaran para salvaguardar la información y mantener un sistema coherente.

\subsection{Sockets}

Los sockets hacen posible la comunicación entre conectores de procesos, cada uno de ellos tiene asociada una dirección, un puerto, y un protocolo asociado (UDP o TCP). Los sockets nos permiten implementar una arquitectura cliente-servidor. La comunicación debe ser iniciada por uno de los procesos que se denomina programa "cliente". El segundo proceso espera a que otro inicie la comunicación, por este motivo se denomina programa "servidor". Un socket es un proceso o hilo existente en la máquina cliente y en la máquina servidora, que sirve en última instancia para que el programa servidor y el cliente lean y escriban la información. Esta información será la transmitida por las diferentes capas de red.\\
He usado los sockets junto a los protocolos mencionados en esta misma sección para la comunicación entre el servidor y el cliente (dispositivo Android).



\section{Librerías}

\subsection{Sqlalchemy}

SQLAlchemy es el conjunto de herramientas SQL de Python y el mapeador relacional de objetos que ofrece a los desarrolladores de aplicaciones toda la potencia y flexibilidad de SQL. Proporciona un conjunto completo de patrones de persistencia de nivel empresarial bien conocidos, diseñados para un acceso a la base de datos eficiente y de alto rendimiento, adaptados a un lenguaje de dominio sencillo y pitónico.

\subsection{OpenCV}

OpenCV es una enorme biblioteca de código abierto para la visión por ordenador, el aprendizaje automático y el procesamiento de imágenes. OpenCV es compatible con una gran variedad de lenguajes de programación como Python, C++, Java, entre otros.\\
OpenCV me ha permitido mantener una conexión con las cámaras disponibles y a su vez obtener las imágenes de estas.

\subsection{SQLite}

SQLite es una biblioteca en lenguaje C que implementa un motor de base de datos SQL pequeño, rápido, autónomo, de alta fiabilidad y con todas las funciones. SQLite es el motor de base de datos más utilizado en el mundo. SQLite está integrado en todos los teléfonos móviles y en la mayoría de los ordenadores, y viene incluido en innumerables aplicaciones que la gente utiliza a diario. SQLite es la API para utilizar una base de datos en Android.

\subsection{Android Support Library}

El paquete Android Support Library es un conjunto de bibliotecas de código que proporcionan versiones compatibles con versiones anteriores de las API del marco de trabajo de Android, así como características que sólo están disponibles a través de las API de la biblioteca. Cada biblioteca de apoyo es compatible con un nivel específico de la API de Android.

\subsection{AndroidX}

AndroidX es el proyecto de código abierto que el equipo de Android utiliza para desarrollar, probar, empaquetar, versionar y liberar bibliotecas dentro de Jetpack. AndroidX es una importante mejora de la biblioteca de soporte original de Android.

\subsection{JavaFx}

JavaFX es un conjunto de paquetes gráficos y multimedia que permite a los desarrolladores diseñar, crear, probar, depurar e implantar aplicaciones cliente enriquecidas que funcionan de forma coherente en diversas plataformas. El SDK de JavaFX-Android contiene una implementación de JavaFX que se ejecuta en Android, junto con algunas herramientas para construir los paquetes de Android.

\subsection{JUnit}

Unit es un marco de pruebas unitarias para el lenguaje de programación Java. Desempeña un papel crucial en el desarrollo dirigido por pruebas, y es una familia de marcos de pruebas unitarias conocidos colectivamente como xUnit. JUnit es un framework de "Unit Testing" para aplicaciones Java que ya está incluido por defecto en android studio.

\subsection{Material Design}

Material Design es un lenguaje de diseño orientado a Android creado por Google, que admite experiencias táctiles en pantalla a través de funciones ricas en pistas y movimientos naturales que imitan los objetos del mundo real. Material design es una guía completa para el diseño visual, de movimiento y de interacción en todas las plataformas y dispositivos.


