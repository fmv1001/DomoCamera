\capitulo{4}{Técnicas y herramientas}

\section{Herramientas de control de versiones}

\subsection{Github}

GitHub es una plataforma que nos permite gestionar y organizar nuestros proyectos y está basada en la nube, incorporando las funciones de control de versiones que proporciona Git. Esta herramienta posibilita a los desarrolladores poder almacenar y administrar su código, realizar un registro y control de los cambios sobre el código almacenado. Es una de las herramientas más populares entre los desarrolladores, cuenta con más de 100 millones de repositorios, y la mayoría de ellos son de código abierto.\\
He utilizado GitHub para el alojamiento de mi proyecto en el repositorio ``https://github.com/fmv1001/LocalStream''.

\subsubsection{Git Bash}

Git Bash es una aplicación para Windows para la emulación de la línea de comandos de Git a través de una shell. Una shell es un intérprete de comandos que provee una interfaz de usuario para la comunicación con el sistema operativo. \\
Esta aplicación me ha servido para subir cambios al respositorio de GitHub a través de los comandos git.



\section{Herramientas de gestión de proyectos}

\subsection{ZenHub}

ZenHub es una plataforma para la gestión ágil de proyectos que se integra con github, funcionando como aplicación nativa en su interfaz. Te ayuda a planificar tu proyecto dentro de GitHub, automatiza el flujo de trabajo. Más del 75\% de los desarrolladores que usan ZenHub en sus proyectos dicen que ZenHub mejora su enfoque y ayuda en el envío de un mejor software en un menor periodo de tiempo, y el 65\% informan de proyectos con un mejor alcance.\\
ZenHub me ha permitido gestionar mi proyecto, ayudándome con la planificación del mismo, para cumplir los tiempos de entrega del mismo.



\section{Metodologías}

\subsection{Desarrollo por fases}

El desarrollo por fases o etapas es un modelo de software en el que las especificaciones no son conocidas desde el inicio, se van desarrollando simultáneamente en las distintas versiones del código, esta metodología es la que se ha llevado a cabo en este proyecto.



\section{Patrones de diseño}

\subsection{Arquitectura Cliente-Servidor}

La arquitectura cliente-servidor es un modelo de diseño de software en una red informática en la que se clasifica a los dispositivos implicados. Un cliente que solicita información (demandante) y el servidor se la proporciona (proveedor de recursos).

\subsection{Patrón Modelo-Vista-Controlador (MVC)}

El patrón MVC es un patrón arquitectónico que separa el sistema en tres capas, los datos, la lógica de la aplicación y la interfaz de usuario. El modelo es la capa de datos, la vista es la interfaz de usuario y el controlador, la lógica de la aplicación.



\section{Herramientas de evaluación de código}

\subsection{SonarQube}

SonarQube es una plataforma de código abierto desarrollada en Java que nos permite realizar análisis de código con diferentes herramientas de forma automatizada. Usa diferentes herramientas de análisis estático de código fuente como pueden ser Checkstyle, PMD o FindBugs para obtener métricas que pueden ayudar a mejorar la calidad del código de un programa.\\
En este proyecto se ha usado la herramienta Sonar-Scanner de SonarQube, para la evaluación del código tanto de la aplicación Android en java como del servidor desarrollado en python.



\section{Herramientas de documentación}

\subsection{\LaTeX}

\LaTeX es un sistema de software libre de composición tipográfica de alta calidad, orientado a la producción de documentación técnica y científica. \LaTeX es el estándar de facto para la comunicación y publicación de documentos científicos gracias a sus características, posibilidades y calidad profesional.\\
He usado \LaTeX para el desarrollo de tanto este documento como de los anexos.

\subsection{TEXMAKER}

Texmaker es un editor \LaTeX, avanzado y multiplataforma. 
Integra las diferentes herramientas necesarias para desarrollar documentos con \LaTeX, todo ello en una sola aplicación.
Incluye compatibilidad para unicode, autocompletado, corrección ortográfica, plegado de código, modo de vista continua y tiene integrado un visor de pdf.
Texmaker es fácil de usar y de configurar, y está publicado bajo la licencia GPL (open source).

\subsection{Draw.io}

\textit{Draw.io} es una herramienta de diagramación que nos permite realizar cualquier tipo de diagramas (diagrmas de flujo, de secuencia, de red, UML, etc). Es muy interesante ya que proporciona elementos UML, muy necesarios en proyectos software.



\section{Lenguajes de programación}

\subsection{Python}

Python es un lenguaje de programación ampliamente utilizado por empresas de casi todo el mundo para la construcción de aplicaciones web, análisis de datos, automatización de operaciones y creación de aplicaciones empresariales con alta fiabilidad y escalabilidad.\\
Para desarrollar el servidor he utilizado la versión 3.0 del lenguaje Python.

\subsection{Java}

Java es un lenguaje de programación orientado a objetos y basado en clases que se usa para el desarrollo de aplicaciones, diseñado para tener la menor cantidad de dependencias de implementación posibles. Además es uno de los lenguajes de programación utilizados para el desarrollo de aplicaciones para el sistema operativo Android. El código compilado de Java puede ejecutarse en cualquier plataforma que tenga instalada la máquina virtual java, sin la necesidad de volver a ser compilado.\\
Java es el lenguaje que he escogido para la construcción de la aplicación Android debido a mi familiaridad con este lenguaje.

\subsection{SQL}

SQL (Structured Query Language) es un lenguaje declarativo que ha sido diseñado para la administración y recuperación de información de sistemas de gestión de bases de datos relacionales. Este lenguaje es conservado por el organismo ANSI (American National Standards Institute).\\
El lenguaje SQL ha sido muy útil en este proyecto a la hora de respaldar al información necesaria tanto en el servidor como en la aplicación.



\section{Entornos de desarrollo integrado (IDE)}

\subsection{Android Studio}

Android Studio es el entorno de desarrollo integrado (IDE) oficial para el desarrollo de aplicaciones para el sistema operativo Android y está basado en IntelliJ IDEA (IDE de Java desarrollado por JetBrains). Además del potente editor de códigos y las herramientas para desarrolladores de IntelliJ, Android Studio ofrece incluso más funciones que aumentan tu productividad cuando desarrollas apps para Android, entre otras tenemos: un sistema de compilación flexible basado en Gradle o la integración con GitHub y plantillas de código para ayudarte a compilar funciones de apps comunes y también importar código de muestra.\\
Android Studio es el IDE oficial para desarrollar aplicaciones en Andorid y por eso lo he escogido para mi proyecto.

\subsection{Visual Studio Code}

Visual Studio Code es un entorno de desarrollo desplegado por Microsoft para todas las plataformas (Windows, Linux y macOS). Incluye soporte para depuración, resaltado de sintaxis, control integrado de Git, finalización inteligente de código, fragmentos y refactorización de código. Visual Studio Code es un proyecto open source aunque la descarga de forma oficial es bajo software privativo de Microsoft incluyendo algunas de sus características personalizadas.\\ 
Para la fase de desarrollo del servidor he utilizado esta aplicación open source de Microsoft.



\section{Herramientas de automatización de compilación del código}

\subsection{Gradle}

Gradle, es una herramienta que permite la automatización de compilación de código abierto, la cual se encuentra centrada en la flexibilidad y el rendimiento. Los scripts de compilación de Gradle se escriben utilizando Groovy o Kotlin DSL (Domain Specific Language). Gradle tiene una gran flexibilidad y nos deja hacer usos otros lenguajes y no solo de Java, también cuenta con un sistema de gestión de dependencias muy estable. Además es el sistema de compilación oficial para Android y cuenta con soporte para diversas tecnologías y lenguajes.



\section{Protocolos de comunicación}

\subsection{UDP}

El protocolo de datagramas de usuario (User Datagram Protocol o UDP) es un protocolo de la capa de transporte basado en el intercambio de datagramas (un datagrama es un paquete de datos que compone el mínimo bloque de información en una red). Permite el envío de datagramas a través de la red sin que se haya establecido previamente una conexión, gracias a la incorporación suficiente de información de direccionamiento en su cabecera. Una de sus ventajas es que no implementa acuse de recibo.\\
Gracias a este protocolo el servidor envía las imágenes al dispositivo móvil Android sin que este confirme que lo ha recibido (no implementa acuse de recibo), perdiendo rendimiento, y el envío es no bloqueante para el hilo que envía.

\subsection{TCP / IP}

TCP es un protocolo del nivel de transporte empleado por aplicaciones que requieren entrega garantizada (acuse de recibo). Se trata de un protocolo que implementa control de flujo para gestionar tiempos de espera, retransmisiones, entre otros. TCP establece un enlace virtual full duplex entre dos puntos finales. Cada punto final está formado por una dirección IP y un puerto TCP.\\
Para que el servidor y la aplicación se comuniquen a la hora de realizar alguna tarea, este protocolo ha permitido que en ningún momento se pierda la conexión y en ese caso los dos serán conscientes y actuaran para salvaguardar la información y mantener un sistema coherente.

\subsection{Sockets}

Los sockets hacen posible la comunicación entre conectores de procesos, cada uno de ellos tiene asociada una dirección, un puerto, y un protocolo asociado (UDP o TCP). Los sockets nos permiten implementar una arquitectura cliente-servidor. La comunicación debe ser iniciada por uno de los procesos que se denomina programa ``cliente''. El segundo proceso espera a que otro inicie la comunicación, por este motivo se denomina programa ``servidor''. Un socket es un proceso o hilo existente en la máquina cliente y en la máquina servidora, que sirve en última instancia para que el programa servidor y el cliente lean y escriban la información. Esta información será la transmitida por las diferentes capas de red.\\
He usado los sockets junto a los protocolos mencionados en esta misma sección para la comunicación entre el servidor y el cliente (dispositivo Android).

\subsection{RTSP}

El Protocolo de Transmisión en Tiempo Real (Real Time Streaming Protocol o RTSP) es una tecnología de vídeo de probada eficacia. Es un protocolo usado en la capa de aplicación para la transferencia de datos de medios de vídeo en tiempo real. Permite el control en la transmisión de audio y vídeo entre puntos finales. También facilita el envío de contenidos en streaming de baja latencia vía Internet. El puerto que usa por defecto RTSP es el 554.\\
En este proyecto se ha usado este protocolo para establecer la conexión con las cámaras, a través de RTSP OpenCV obtiene el control.



\section{Librerías}

\subsection{SQLAlchemy}

SQLAlchemy es el conjunto de herramientas SQL para Python además de ser el mapeador relacional de objetos que ofrece toda la potencia y flexibilidad del lenguaje SQL. Aporta un complejo conjunto de patrones de persistencia muy conocidos, diseñados para un acceso eficiente y de alto rendimiento a la base de datos, adaptados a un lenguaje pitónico y sencillo.

\subsection{OpenCV}

OpenCV es una biblioteca de código abierto para visión artificial, aprendizaje automático y procesamiento de imágenes. Es compatible con gran variedad de lenguajes de programación como Python, C++, Java, entre otros.\\
OpenCV me ha permitido mantener una conexión con las cámaras disponibles y a su vez obtener las imágenes de estas.

\subsection{SQLite}

SQLite es una biblioteca que hace uso de lenguaje C para implementar un pequeño motor de base de datos SQL haciendo que sea rápido, autónomo, manteniendo una alta fiabilidad y con todas las funciones necesarias para una base de datos. SQLite es el motor de base de datos más utilizado en el mundo. La biblioteca \textit{SQLite} está incluida en todos los teléfonos móviles además de en la mayoría de los ordenadores, y viene integrado en un número inmenso de aplicaciones de uso diario. Además, SQLite es la API para utilizar una base de datos en Android.

\subsection{Android Support Library}

El paquete Android Support Library es un conjunto de bibliotecas que proporcionan versiones compatibilidad con versiones anteriores de APIs de la estructura Android, así como algunas características que únicamente están disponibles por medio de las API de esta biblioteca. Cada biblioteca del paquete es compatible con una versión específica de la API de Android.

\subsection{AndroidX}

AndroidX es el proyecto de open source que Android utiliza para desarrollar, probar, empaquetar, versionar y liberar bibliotecas dentro de Jetpack. AndroidX ha mejorado y sustituido al paquete Android Support Library.

\subsection{JavaFx}

JavaFX a través de paquetes gráficos y multimedia nos permite diseñar, crear, probar, depurar e implantar apps clientes que funcionaran de coherentemente en las distintas plataformas. El SDK de JavaFX-Android posee una implementación de JavaFX para ser ejecutado en Android, junto con otras herramientas para la construcción de paquetes en Android.

\subsection{JCodec}

JCodec es un librería Java que implementa códecs de vídeo y audio. Actualmente JCodec tiene soporte en Android, y es muy útil si manejas APIs Android de versiones antiguas. Esta librería me ha permitido poder crear un archivo de vídeo sobre las imágenes que llegan desde el servidor.

\subsection{Material Design}

Material Design es un lenguaje de diseño creado por Google orientado a Android, que admite movimientos y pulsaciones táctiles en pantalla gracias a funciones y movimientos que imitan objetos del mundo real. Material design es una guía completa para el diseño visual, de movimiento y de interacción en todas las plataformas y dispositivos.


