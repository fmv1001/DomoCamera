\capitulo{1}{Introducción}

Hoy en día en los hogares cada vez estamos más rodeados de tecnologías, tanto móviles, como aparatos del hogar, los cuales disponen de sistemas informáticos muy avanzados y automatizados que nos proporcionan herramientas para ser manejados a distancia. Hablamos de lavadoras, frigoríficos, enchufes, interruptores, termostatos, e incluso aspiradores inteligentes, todos ellos monitorizables a distancia.

La domótica es el conjunto de todas las tecnologías que agrupan el control y la automatización inteligente de una vivienda, aportando servicios de gestión energética, bienestar, comunicación y seguridad. Gracias a ella llegaremos a gestionar de una manera más eficiente nuestros recursos, además de aportarnos más seguridad y un extra de confort.

Pero no sólo ha ayudado a las personas, gracias a la eficiencia energética y de recursos, también colabora con el cambio climático. Gracias a este sector, se reducen los consumos en calefacción, en energía para la vivienda, en gasto de cantidad de agua, etc, todo ello contribuye a detener el cambio climático además de a la población.

Es en la seguridad en lo que me he basado para la realización de este proyecto, crear un entorno que ayude en la gestión de la seguridad en la vivienda. Todo ello gestionado a través de un dispositivo móvil, debido a que hoy en día es indispensable el uso de un dispositivo móvil y todas las tecnologías se están desarrollando en base a ellos, por ejemplo el NFC para pagar sin necesidad de una tarjeta.

Además, el sector domótico está creciendo a gran velocidad hoy en día y ha evolucionado notablemente recientemente. Gracias a esta evolución y crecimiento, podemos obtener más funcionalidad, facilidad de uso e instalación, variedad de producto, además de mayor calidad y mayor oferta.



\section{Estructura de la memoria}

La memoria sigue la siguiente estructura:

\begin{itemize}
\item
  \textbf{Introducción:} Descripción e introducción del proyecto. También podremos encontrar la estructura de la memoria, de los anexos y de los materiales adjuntos.
\item
  \textbf{Objetivos del proyecto:} explicación de los objetivos que se pretenden cumplir. 
\item
  \textbf{Conceptos teóricos:} breve introducción a los conceptos teóricos clave para la comprensión del desarrollo del proyecto.
\item
  \textbf{Técnicas y herramientas:} conjunto de técnicas metodológicas y
  herramientas empleadas durante el desarrollo del proyecto.
\item
  \textbf{Aspectos relevantes del desarrollo del proyecto:} exposición de los aspectos más relevantes en el desarrollo del proyecto.
\item
  \textbf{Trabajos relacionados:} pequeña presentación y comparación con algunos trabajos relacionados con el presente trabajo.
\item
  \textbf{Conclusiones y Líneas de trabajo futuras:} conclusiones derivadas tras la realización del proyecto, así como posibles mejoras futuras del resultado del desarrollo del proyecto.
\end{itemize}

\section{Estructura de los anexos}

Los anexos siguen la siguiente estructura:

\begin{itemize}
\item
  \textbf{Plan del proyecto Software:} desarrollo de la planificación temporal y el estudio de viabilidad del proyecto.
\item
  \textbf{Especificación de Requisitos:} requisitos derivados de los objetivos del proyecto.
\item
  \textbf{Especificación de diseño:} descripción del diseño del sistema con los consiguientes diagramas.
\item
  \textbf{Documentación técnica de programación:} explicación de los recursos necesarios para trabajar con el proyecto (entornos de desarrollo, lenguajes, etc).
\item
  \textbf{Documentación de usuario:} guía para usuario final, en ella se expone como se debe usar el producto final.
\end{itemize}

\section{Materiales adjuntos}

\subsection{Contenido del CD}

El CD que se ha entregado junto con la memoria se estructura de la siguiente manera:

\begin{itemize}
\item
	\textbf{Memoria:} se incluye la última versión de la memoria en formato pdf.
\item
	\textbf{Anexos:} se incluye la última versión de los anexos en formato pdf.
\item
	\textbf{Código:} se incluye la última versión del código tanto de la aplicación como del servidor.
\item
	\textbf{Vídeo:} se incluye una demo vídeo del funcionamiento del producto final.
\end{itemize}

\subsection{Enlaces de los materiales del proyecto}

\begin{itemize}
\item
	\href{https://github.com/fmv1001/DomoCamera}{Repositorio del proyecto}.
\item
	\href{https://youtu.be/txAF8uAJ6B0}{Vídeo demostración} (enlace a YouTube).
\end{itemize}