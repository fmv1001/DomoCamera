\capitulo{1}{Introducción}

Hoy en día en los hogares cada vez estamos más rodeados de tecnologías, tanto móviles, como aparatos del hogar, los cuales disponen de sistemas informáticos muy avanzados y automatizados que nos proporcionan herramientas para ser manejados a distancia. Hablamos de lavadoras, frigoríficos, enchufes, interruptores, termostatos, e incluso aspiradores inteligentes, todos ellos monitorizables a distancia.\\

La domótica es el conjunto de todas las tecnologías que agrupan el control y la automatización inteligente de una vivienda. Gracias a ella llegaremos a gestionar de una manera más eficiente nuestros recursos, además de aportarnos más seguridad y un extra de confort.\\

El sector domótico está creciendo a gran velocidad hoy en día y ha evolucionado notablemente recientemente. Gracias a esta evolución y crecimiento, podemos obtener más funcionalidad, facilidad de uso e instalación, variedad de producto, además de mayor calidad y mayor oferta.\\

Pero no sólo ha ayudado a las personas, gracias a la eficiencia energética y de recursos, también colabora con el cambio climático. Gracias a este sector, se reducen los consumos en calefacción, en energía para la vivienda, en gasto de cantidad de agua, etc, todo ello contribuye a detener el cambio climático además de a la población.
En resumen, la domótica ha ayudado a mejorar la calidad de vida de las personas.
