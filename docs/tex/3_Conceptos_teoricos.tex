\capitulo{3}{Conceptos teóricos}

Para comprender el marco teórico del desarrollo de este proyecto, debemos conocer previamente algunos conceptos en los que se basa el mismo.

\section{Domótica}

La domótica \cite{domoticaCEDOM} según la \href{http://www.cedom.es/}{Asociación Española de Domótica e Inmótica} (CEDOM) es el conjunto de tecnologías aplicadas al control y la automatización inteligente de la vivienda, que permite una gestión eficiente del uso de la energía, que aporta seguridad y confort, además de comunicación entre el usuario y el sistema. La domótica aporta servicios de gestión energética, seguridad, bienestar y comunicación.\\

En este proyecto se centra en dar un servicio de seguridad en el que vamos a controlar cámaras en un vivienda y en futuras versiones podrían añadirse otros elementos inteligentes del hogar. 



\section{Android}

Android \cite{book:android} es un sistema operativo (OS) que inicialmente fue desarrollado para dispositivos móviles pero que hoy en día engloba ordenadores, relojes, televisores, tablets, coches, etc. Este entorno ha sido desarrollado por Google cuyo obetivo fue fomentar el uso de una plataforma abierta, gratuita, multiplataforma y muy segura, y es por ello que está basado en Linux (Linux es un núcleo sistema operativo (OS) open source).\\
Este sistema permite programar aplicaciones empleando una variación de Java llamada Dalvik (o ART a partir de su version 5.0) y proporciona todas las interfaces necesarias para desarrollar fácilmente aplicaciones que acceden a las funciones del dispositivo (como pudiera ser el GPS, o la memoria entre otros).\\
Y es por todo esto que es el sistema operativo más usado el mundo.

\imagen{os_combined_2009_to_2021}{Sistemas operativos mas usados desde 2009 a 2021 \cite{androidvsworld}}



\section{Servidor}

Un servidor es una máquina que está integrada en una red informática que se encarga de realizar tareas para los clientes. Su funcionamiento se basa en el modelo Cliente-Servidor (figura \ref{fig:cliente-servidor-img}), un servidor provee un servicio y los clientes solicitan recursos mediante peticiones y el servidor provee los recursos con respuestas.

\imagen{cliente-servidor-img}{Arquitectura Cliente-Servidor}

Existen muchos tipos de servidores, tanto servidores web, como FTP, proxy o inlcuso de juegos online. En este proyecto se va a desarrollar un servidor proveedor de servicios, en este caso monitorización de cámaras.



\section{Python}

Python \cite{python} es un lenguaje interpretado, interactivo y orientado a objetos, que incorpora módulos, excepciones, tipado dinámico, tipos de datos de muy alto nivel y clases. Está muy extendido y nos permite realizar cualquiera de nuestros propósitos gracias a su gran cantidad de librerías y programadores. \\
Fue creado a principios de la década de 1990 por Guido van Rossum en Stichting Mathematisch Centrum (CWI) en los Países Bajos como sucesor de un idioma llamado ABC \cite{pythonhistory}. \\
Este lenguaje \cite{python} nos proporciona una gran biblioteca estándar que abarca áreas como procesamiento de cadenas (ya sea tanto expresiones regulares, Unicode, o incluso cálculo de diferencias entre archivos), protocolos de Internet (como son HTTP, FTP, SMTP, XML-RPC, POP, IMAP,o programación CGI entre otros), ingeniería de software (desde pruebas unitarias, registro, creación de perfiles, hasta análisis del propio código Python) e interfaces del sistema operativo (tales como llamadas al sistema, sistemas de archivos, incluidos sockets TCP / IP).
\imagen{growth_major_languages}{lenguajes de programación mas usados desde 2012 a 2018 \cite{pythonuse}.}



\section{Dispositivos de bajo consumo}

Hoy en día los dispositivos de bajo consumo están en auge.
Con las nuevas normativas debido al cambio climático las empresas están dedicando sus recursos a diseñar dispositivos de bajo consumo. 
En este proyecto pensé en la Raspberry Pi ya que tengo una en casa y he realizado algún proyecto ya con ella.

\subsection{Raspberry Pi}

La Raspberry Pi \cite{raspPi} es un ordenador de bajo coste, del tamaño de una tarjeta de crédito. Es un pequeño dispositivo capaz de permitir a personas de todas las edades explorar la informática y aprender a programar en lenguajes como Scratch y Python. Es capaz de hacer todo lo que se espera de un ordenador de sobremesa, desde navegar por Internet y reproducir vídeo de alta definición, hasta hacer hojas de cálculo, procesar textos y jugar. La Raspberry Pi, sin ventilador y de bajo consumo, funciona de forma muy silenciosa y consume mucha menos energía que otros ordenadores.\\

En artículo \textit{``Power consumption of the Raspberry Pi: A comparative analysis''} \cite{art:raspicomsuptiom} compara los distintos consumos que puede realizar los distintos dispositivos respecto de una Raspberry Pi, la cual es mucho más eficiente que un PC de sobremesa en el orden de consumir hasta 10 veces menos.

\imagen{piconsumo}{Comparación de consumo en vatios (watts) de diferentes dispositivos en distintas pruebas.}

La Raspberry Pi es un proyecto de la \href{https://www.raspberrypi.org/about/}{\textit{Raspberry Pi Foundation}}.\\










