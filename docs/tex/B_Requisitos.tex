\apendice{Especificación de Requisitos}

\section{Introducción}

La Especificación de Requisitos Software (ERS) pretende obtener una descripción completa del comportamiento del sistema que va a ser desarrollado. 
En él se anotan las necesidades del producto (tanto para usuario, como para cliente), además de definirse los requerimientos del sistema. 
Incluye la totalidad de los casos de uso de las interacciones del usuario final con el software.
Este documento sirve de medio de comunicación entre todas las partes (desarrolladores, clientes y usuarios finales).\\

Según el estándar IEEE-STD-830-1998 \cite{ieee830} un ERS debe presentar las siguientes características:

\begin{itemize}
\item
	\textbf{Correcto}\\
	Un ERS es correcto si, y sólo si, cada requisito declarado se encuentra en el software
\item
	\textbf{Inequívoco}\\
	Un SRS es inequívoco si, y sólo si, cada requisito declarado tiene una sola interpretación.
\item
	\textbf{Completo}\\
	Un SRS es completo si, y sólo si, incluye: 
		\begin{itemize}
		\item
			Requisitos relacionados a funcionalidad, desarrollo, restricciones
de diseño, atributos e interfaces externas.
		\item
			 Definiciones de las respuestas del software a todos los posibles datos de la entrada del sistema y a toda clase de situaciones.
		\item
			 Tener todas las etiquetas llenas y bien referencias. 
		\end{itemize}
\item
	\textbf{Consistente}\\
	Un SRS debe ser coherente con los requerimientos del ERS y con todos los documentos de distinto nivel.
\item
	\textbf{Delinear que tiene importancia y/o estabilidad}\\
	Un SRS debe delinear la importancia y/o estabilidad de cada requisito. Cada requisito mantiene un identificador que identifica la importancia o estabilidad de dicho requisito. 
\item
	\textbf{Comprobable}\\
	Un SRS es comprobable si, y sólo si, cada requisito declarado es verificable. 
	Un requisito es verificable si, y sólo si, allí existe algún método finito para verificar que el producto del software reúne el requisito.
\item
	\textbf{Modificable}\\
	Un SRS es modificable si, y sólo si, su estructura y estilo permite cualquier cambio a los requisitos fácilmente, completamente y de forma consistente mientras conserva la estructura y estilo.
\item
	\textbf{Identificable}\\
	Un SRS es identificable si el origen de los requisitos es claro y facilita las referencias en el desarrollo futuro o documentación del mismo. 
\end{itemize}

\section{Objetivos generales}

Se pretende cumplir los objetivos siguientes:

\begin{itemize}
\tightlist
\item
  Desarrollar una aplicación para el entorno Android que permita la monitorización de cámaras dispuestas en un hogar.
\item
  Desarrollar y desplegar un servidor que gestione el entorno domótico (la conexión con las cámaras y con la aplicación Android).
\item
  Facilitar al usuario la monitorización mediante una interfaz fácil y sencilla.
\item
  Utilizar dispositivos de bajo consumo.
\end{itemize}



\section{Catálogo de requisitos}

En este apartado se presentarán los requisitos del sistema, tanto los funcionales como los no funcionales.

\subsection{Requisitos funcionales}

\begin{itemize}
	\item 
		\textbf{RF-1 Gestión de cámaras:} 
			El sistema debe de ser capaz de realizar la gestión de las cámaras.
		\begin{itemize}
			\item \label{RF1-1}
				\textbf{RF-1.1 Añadir cámara:}
					El usuario debe de ser capaz de añadir una cámara con un nombre que la identifique, la ip de acceso a la cámara y un puerto de envío. 
					\begin{itemize}
						\item
							\textbf{RF-1.1.1 Conectar cámara:}
								El servidor debe de ser capaz de conectarse a una cámara a petición del usuario.
					\end{itemize}
			\item \label{RF1-2}
				\textbf{RF-1.2 Eliminar cámara:}
					El usuario debe de ser capaz de eliminar una cámara existente en el sistema.
		\end{itemize}
		
	\item
		\textbf{RF-2 Gestión de la conexión aplicación-servidor:} 
			El sistema debe de ser capaz de realizar la gestión de la conexión del servidor con la aplicación.
		\begin{itemize}
			\item \label{RF2-1}
				\textbf{RF-2.1 Conectar al servidor:}
					El usuario debe de ser capaz de conectarse al servidor a través de la ip del servidor y el puerto de escucha. 
			\item \label{RF2-2}
				\textbf{RF-2.2 Desconectar del servidor:}
					El usuario debe de ser capaz de desconectarse del servidor.
		\end{itemize}
		
	\item
		\textbf{RF-3 Monitorización de las cámaras:} 
			El usuario debe ser capaz de poder monitorizar todas las cámaras conectadas al sistema.
			\begin{itemize}
			\item \label{RF3-1}
				\textbf{RF-3.1 Visualizar imagen:}
					El usuario debe de ser capaz de visualizar la imagen procedente de las cámaras.
			\item
				\textbf{RF-3.2 Grabar un vídeo de una cámara:} 
					El usuario debe de ser capaz de poder realizar una pequeña grabación de la imagen que suministra el servidor.
		\end{itemize}
	
	\item
		\textbf{RF-4 Gestión del servidor:} 
			El sistema debe de ser capaz de realizar la gestión del servidor.
		\begin{itemize}
			\item
				\textbf{RF-4.1 Iniciar servidor:}
					El usuario debe de ser capaz de iniciar el servidor.
			\item \label{RF4-2}
				\textbf{RF-4.2 Parar servidor:}
					El usuario debe de ser capaz de parar el servidor.
		\end{itemize}
	\item
		\textbf{RF-5 Gestión del log del sistema:} 
			La aplicación debe ser capaz de mantener un log del sistema.
			\begin{itemize}
			\item
				\textbf{RF-5.1 Visualizar del log del sistema:}
					El usuario debe poder visualizar un log donde se muestren tanto las acciones realizadas como los posibles errores que sucedan durante el uso de la misma.
		\end{itemize}
\end{itemize}



\subsection{Requisitos no funcionales}

\begin{itemize}
	\item
		\textbf{RNF-1 Usabilidad:}
			La aplicación debe mantener una interfaz de usuario intuitiva, debe parecerse lo suficiente a otras aplicaciones del mismo entorno para que la curva de aprendizaje sea lo suficientemente corta como para la no necesidad de un tutorial.
	\item
		\textbf{RNF-2 Rendimiento:} 
			El sistema debe mantener una cierta fluidez, no se deben presentar fallos o paradas del sistema, independientemente del dispositivo utilizado.
	\item
		\textbf{RNF-3 Disponibilidad:} 
			El servidor debe ser accesible en cualquier momento desde la aplicación mientras este haya sido iniciado. 
			Además la aplicación debe estar disponible para su uso independientemente del momento de acceso.
	\item
		\textbf{RNF-4 Estabilidad:} 
			El sistema debe ser estable, debe mantener un nivel bajo de fallos y ser capaz de enmascarar los mismos al usuario.
	\item
		\textbf{RNF-5 Robustez:} 
			El sistema debe ser capaz de soportar circunstancias no anticipadas sin que se produzca una caída o fallo del sistema.
	\item
		\textbf{RNF-6 Fiabilidad (validez e integridad):} 
			El sistema debe mantener la validez de todas las acciones que se realizan en el mismo, mantenido la integridad de los datos.
	\item
		\textbf{RNF-7 Seguridad:} 
			El sistema debe mantener la seguridad de forma que no se comprometa la información del usuario.
	\item
		\textbf{RNF-8 Mantenibilidad:} 
			El sistema debe permitir la corrección de fallos, mejora de rendimiento (escalabilidad) o adaptaciones con facilidad.
	\item
		\textbf{RNF-9 Soporte:} 
			La aplicación debe dar soporte a dispositivos con versiones de Android iguales o superiores a Android 4.1 (Jelly Bean).
\end{itemize}



\section{Especificación de requisitos}

En este apartado vamos a mostrar los casos de uso de cada una de las acciones derivadas de los requisitos funcionales del proyecto.

\subsection{Caso de uso general}

\imagen{casoUsoGeneral}{Caso de uso general de la especificación de requisitos.}

\subsection{Actores}

El único  actor que actúa con el sistema es el usuario final.

\subsection{Casos de uso}

%CASO DE USO 1

\begin{longtable}[h!]{@{}ll@{}}
\toprule
\begin{minipage}[b]{0.23\columnwidth}\raggedright\strut
\textbf{CU-1}\strut
\end{minipage} & \begin{minipage}[b]{0.71\columnwidth}\raggedright\strut
\textbf{Gestión de cámaras}\strut
\end{minipage}\tabularnewline
\midrule
\endhead
\begin{minipage}[t]{0.23\columnwidth}\raggedright\strut
\textbf{Versión}\strut
\end{minipage} & \begin{minipage}[t]{0.71\columnwidth}\raggedright\strut
1.0\strut
\end{minipage}\tabularnewline
\begin{minipage}[t]{0.23\columnwidth}\raggedright\strut
\textbf{Autor}\strut
\end{minipage} & \begin{minipage}[t]{0.71\columnwidth}\raggedright\strut
\nombre\strut
\end{minipage}\tabularnewline
\begin{minipage}[t]{0.23\columnwidth}\raggedright\strut
\textbf{Requisitos asociados}\strut
\end{minipage} & \begin{minipage}[t]{0.71\columnwidth}\raggedright\strut
RF-1, RF-1.1, RF-1.1.1, RF-1.2\strut
\end{minipage}\tabularnewline
\begin{minipage}[t]{0.23\columnwidth}\raggedright\strut
\textbf{Descripción}\strut
\end{minipage} & \begin{minipage}[t]{0.71\columnwidth}\raggedright\strut
Permite al usuario gestionar sus cámaras.\strut
\end{minipage}\tabularnewline
\begin{minipage}[t]{0.23\columnwidth}\raggedright\strut
\textbf{Precondición}\strut
\end{minipage} & \begin{minipage}[t]{0.71\columnwidth}\raggedright\strut
El usuario debe encontrarse en la pestaña principal (home).\\
La aplicación debe estar conectada al servidor.\strut
\end{minipage}\tabularnewline
\begin{minipage}[t]{0.23\columnwidth}\raggedright\strut
\textbf{Acciones}\strut
\end{minipage} & \begin{minipage}[t]{0.71\columnwidth}\raggedright\strut
\begin{enumerate}
\def\labelenumi{\arabic{enumi}.}
\tightlist
\item
  Se listan todas las cámaras que están disponibles.
\item
  Para cada una de las cámaras se muestra un botón para poder acceder a ella.
\item
  Se muestra un menú en el que al pulsar podremos añadir o eliminar cámaras.
\end{enumerate}\strut
\end{minipage}\tabularnewline
\begin{minipage}[t]{0.23\columnwidth}\raggedright\strut
\textbf{Postcondición}\strut
\end{minipage} & \begin{minipage}[t]{0.71\columnwidth}\raggedright\strut
El número de cámaras mostrado es el mismo que el servidor y la aplicación tienen en sus bases de datos.\strut
\end{minipage}\tabularnewline
\begin{minipage}[t]{0.23\columnwidth}\raggedright\strut
\textbf{Excepciones}\strut
\end{minipage} & \begin{minipage}[t]{0.71\columnwidth}\raggedright\strut
\begin{itemize}
\tightlist
\item
  Error al cargar cámaras (mensaje).
\item
  No hay cámaras introducidas (pantalla vacía).
\end{itemize}\strut
\end{minipage}\tabularnewline
\begin{minipage}[t]{0.23\columnwidth}\raggedright\strut
\textbf{Importancia}\strut
\end{minipage} & \begin{minipage}[t]{0.71\columnwidth}\raggedright\strut
Alta\strut
\end{minipage}\tabularnewline
\bottomrule
\caption{CU-1 Gestión de cámaras.}
\end{longtable}

%CASO DE USO 2 ---------------------------------------------------------

\begin{longtable}[h!]{@{}ll@{}}
\toprule
\begin{minipage}[b]{0.23\columnwidth}\raggedright\strut
\textbf{CU-2}\strut
\end{minipage} & \begin{minipage}[b]{0.71\columnwidth}\raggedright\strut
\textbf{Añadir cámara}\strut
\end{minipage}\tabularnewline
\midrule
\endhead
\begin{minipage}[t]{0.23\columnwidth}\raggedright\strut
\textbf{Versión}\strut
\end{minipage} & \begin{minipage}[t]{0.71\columnwidth}\raggedright\strut
1.0\strut
\end{minipage}\tabularnewline
\begin{minipage}[t]{0.23\columnwidth}\raggedright\strut
\textbf{Autor}\strut
\end{minipage} & \begin{minipage}[t]{0.71\columnwidth}\raggedright\strut
\nombre\strut
\end{minipage}\tabularnewline
\begin{minipage}[t]{0.23\columnwidth}\raggedright\strut
\textbf{Requisitos asociados}\strut
\end{minipage} & \begin{minipage}[t]{0.71\columnwidth}\raggedright\strut
RF-1.1, RF-1.1.1\strut
\end{minipage}\tabularnewline
\begin{minipage}[t]{0.23\columnwidth}\raggedright\strut
\textbf{Descripción}\strut
\end{minipage} & \begin{minipage}[t]{0.71\columnwidth}\raggedright\strut
Permite al usuario añadir una cámara.\strut
\end{minipage}\tabularnewline
\begin{minipage}[t]{0.23\columnwidth}\raggedright\strut
\textbf{Precondición}\strut
\end{minipage} & \begin{minipage}[t]{0.71\columnwidth}\raggedright\strut
El usuario debe encontrarse en la pestaña principal (home).\\
La aplicación debe estar conectada al servidor.\strut
\end{minipage}\tabularnewline
\begin{minipage}[t]{0.23\columnwidth}\raggedright\strut
\textbf{Acciones}\strut
\end{minipage} & \begin{minipage}[t]{0.71\columnwidth}\raggedright\strut
\begin{enumerate}
\def\labelenumi{\arabic{enumi}.}
\tightlist
\item
  El usuario pulsa el botón del menú desplegable superior derecho.
\item
  El usuario elige la opción ``\textit{Añadir cámara}''.
\item
  El usuario rellena los campos \textit{nombre}, \textit{IP} y \textit{Puerto}.
\item
  El usuario pulsa el botón aceptar.
\end{enumerate}\strut
\end{minipage}\tabularnewline
\begin{minipage}[t]{0.23\columnwidth}\raggedright\strut
\textbf{Postcondición}\strut
\end{minipage} & \begin{minipage}[t]{0.71\columnwidth}\raggedright\strut
Se ha añadido una nueva cámara en la pantalla principal y la información ha sido almacenada en las bases de datos.\strut
\end{minipage}\tabularnewline
\begin{minipage}[t]{0.23\columnwidth}\raggedright\strut
\textbf{Excepciones}\strut
\end{minipage} & \begin{minipage}[t]{0.71\columnwidth}\raggedright\strut
\begin{itemize}
\tightlist
\item
  Error al en la conexión (mensaje).
\item
  Cámara ya existente (mensaje).
\end{itemize}\strut
\end{minipage}\tabularnewline
\begin{minipage}[t]{0.23\columnwidth}\raggedright\strut
\textbf{Importancia}\strut
\end{minipage} & \begin{minipage}[t]{0.71\columnwidth}\raggedright\strut
Alta\strut
\end{minipage}\tabularnewline
\bottomrule
\caption{CU-2 Añadir cámara.}
\end{longtable}

%CASO DE USO 3 ---------------------------------------------------------

\begin{longtable}[h!]{@{}ll@{}}
\toprule
\begin{minipage}[b]{0.23\columnwidth}\raggedright\strut
\textbf{CU-3}\strut
\end{minipage} & \begin{minipage}[b]{0.71\columnwidth}\raggedright\strut
\textbf{Conectar cámara.}\strut
\end{minipage}\tabularnewline
\midrule
\endhead
\begin{minipage}[t]{0.23\columnwidth}\raggedright\strut
\textbf{Versión}\strut
\end{minipage} & \begin{minipage}[t]{0.71\columnwidth}\raggedright\strut
1.0\strut
\end{minipage}\tabularnewline
\begin{minipage}[t]{0.23\columnwidth}\raggedright\strut
\textbf{Autor}\strut
\end{minipage} & \begin{minipage}[t]{0.71\columnwidth}\raggedright\strut
\nombre\strut
\end{minipage}\tabularnewline
\begin{minipage}[t]{0.23\columnwidth}\raggedright\strut
\textbf{Requisitos asociados}\strut
\end{minipage} & \begin{minipage}[t]{0.71\columnwidth}\raggedright\strut
RF-1.1.1\strut
\end{minipage}\tabularnewline
\begin{minipage}[t]{0.23\columnwidth}\raggedright\strut
\textbf{Descripción}\strut
\end{minipage} & \begin{minipage}[t]{0.71\columnwidth}\raggedright\strut
Permite al usuario gestionar sus cámaras.\strut
\end{minipage}\tabularnewline
\begin{minipage}[t]{0.23\columnwidth}\raggedright\strut
\textbf{Precondición}\strut
\end{minipage} & \begin{minipage}[t]{0.71\columnwidth}\raggedright\strut
El usuario debe haber añadido la cámara.\\
La aplicación debe estar conectada al servidor.\strut
\end{minipage}\tabularnewline
\begin{minipage}[t]{0.23\columnwidth}\raggedright\strut
\textbf{Acciones}\strut
\end{minipage} & \begin{minipage}[t]{0.71\columnwidth}\raggedright\strut
\begin{enumerate}
\def\labelenumi{\arabic{enumi}.}
\tightlist
\item
  Se comunica la acción al servidor.
\item
  El servidor realiza la conexión con la cámara.
\item
  Se actualizan las bases de datos.
\end{enumerate}\strut
\end{minipage}\tabularnewline
\begin{minipage}[t]{0.23\columnwidth}\raggedright\strut
\textbf{Postcondición}\strut
\end{minipage} & \begin{minipage}[t]{0.71\columnwidth}\raggedright\strut
Se ha establecido una conexión con la cámara.\strut
\end{minipage}\tabularnewline
\begin{minipage}[t]{0.23\columnwidth}\raggedright\strut
\textbf{Excepciones}\strut
\end{minipage} & \begin{minipage}[t]{0.71\columnwidth}\raggedright\strut
\begin{itemize}
\tightlist
\item
  Error al intentar realizar conexión con cámara.
\end{itemize}\strut
\end{minipage}\tabularnewline
\begin{minipage}[t]{0.23\columnwidth}\raggedright\strut
\textbf{Importancia}\strut
\end{minipage} & \begin{minipage}[t]{0.71\columnwidth}\raggedright\strut
Alta\strut
\end{minipage}\tabularnewline
\bottomrule
\caption{CU-3 Conectar cámara.}
\end{longtable}

%CASO DE USO 4 ---------------------------------------------------------

\begin{longtable}[h!]{@{}ll@{}}
\toprule
\begin{minipage}[b]{0.23\columnwidth}\raggedright\strut
\textbf{CU-4}\strut
\end{minipage} & \begin{minipage}[b]{0.71\columnwidth}\raggedright\strut
\textbf{Eliminar cámara}\strut
\end{minipage}\tabularnewline
\midrule
\endhead
\begin{minipage}[t]{0.23\columnwidth}\raggedright\strut
\textbf{Versión}\strut
\end{minipage} & \begin{minipage}[t]{0.71\columnwidth}\raggedright\strut
1.0\strut
\end{minipage}\tabularnewline
\begin{minipage}[t]{0.23\columnwidth}\raggedright\strut
\textbf{Autor}\strut
\end{minipage} & \begin{minipage}[t]{0.71\columnwidth}\raggedright\strut
\nombre\strut
\end{minipage}\tabularnewline
\begin{minipage}[t]{0.23\columnwidth}\raggedright\strut
\textbf{Requisitos asociados}\strut
\end{minipage} & \begin{minipage}[t]{0.71\columnwidth}\raggedright\strut
RF-1.2\strut
\end{minipage}\tabularnewline
\begin{minipage}[t]{0.23\columnwidth}\raggedright\strut
\textbf{Descripción}\strut
\end{minipage} & \begin{minipage}[t]{0.71\columnwidth}\raggedright\strut
Permite al usuario eliminar una cámara.\strut
\end{minipage}\tabularnewline
\begin{minipage}[t]{0.23\columnwidth}\raggedright\strut
\textbf{Precondición}\strut
\end{minipage} & \begin{minipage}[t]{0.71\columnwidth}\raggedright\strut
El usuario debe encontrarse en la pestaña principal (home).\\
La aplicación debe estar conectada al servidor.\strut
\end{minipage}\tabularnewline
\begin{minipage}[t]{0.23\columnwidth}\raggedright\strut
\textbf{Acciones}\strut
\end{minipage} & \begin{minipage}[t]{0.71\columnwidth}\raggedright\strut
\begin{enumerate}
\def\labelenumi{\arabic{enumi}.}
\tightlist
\item
  El usuario pulsa el botón del menú desplegable superior derecho.
\item
  El usuario elige la opción ``\textit{Eliminar cámara}''.
\item
  El usuario escoge una opción (cámara) de las mostradas.
\item
  El usuario pulsa el botón aceptar.
\end{enumerate}\strut
\end{minipage}\tabularnewline
\begin{minipage}[t]{0.23\columnwidth}\raggedright\strut
\textbf{Postcondición}\strut
\end{minipage} & \begin{minipage}[t]{0.71\columnwidth}\raggedright\strut
Se ha eliminado la cámara de la pantalla principal y la información ha sido actualizada en las bases de datos.\strut
\end{minipage}\tabularnewline
\begin{minipage}[t]{0.23\columnwidth}\raggedright\strut
\textbf{Excepciones}\strut
\end{minipage} & \begin{minipage}[t]{0.71\columnwidth}\raggedright\strut
\begin{itemize}
\tightlist
\item
  Error al en la conexión (mensaje).
\item
  Error al eliminar (mensaje).
\end{itemize}\strut
\end{minipage}\tabularnewline
\begin{minipage}[t]{0.23\columnwidth}\raggedright\strut
\textbf{Importancia}\strut
\end{minipage} & \begin{minipage}[t]{0.71\columnwidth}\raggedright\strut
Alta\strut
\end{minipage}\tabularnewline
\bottomrule
\caption{CU-4 Eliminar cámara.}
\end{longtable}

%CASO DE USO 5 ---------------------------------------------------------

\begin{longtable}[h!]{@{}ll@{}}
\toprule
\begin{minipage}[b]{0.23\columnwidth}\raggedright\strut
\textbf{CU-5}\strut
\end{minipage} & \begin{minipage}[b]{0.71\columnwidth}\raggedright\strut
\textbf{Gestión de la conexión aplicación-servidor}\strut
\end{minipage}\tabularnewline
\midrule
\endhead
\begin{minipage}[t]{0.23\columnwidth}\raggedright\strut
\textbf{Versión}\strut
\end{minipage} & \begin{minipage}[t]{0.71\columnwidth}\raggedright\strut
1.0\strut
\end{minipage}\tabularnewline
\begin{minipage}[t]{0.23\columnwidth}\raggedright\strut
\textbf{Autor}\strut
\end{minipage} & \begin{minipage}[t]{0.71\columnwidth}\raggedright\strut
\nombre\strut
\end{minipage}\tabularnewline
\begin{minipage}[t]{0.23\columnwidth}\raggedright\strut
\textbf{Requisitos asociados}\strut
\end{minipage} & \begin{minipage}[t]{0.71\columnwidth}\raggedright\strut
RF-2, RF-2.1, RF-2.2\strut
\end{minipage}\tabularnewline
\begin{minipage}[t]{0.23\columnwidth}\raggedright\strut
\textbf{Descripción}\strut
\end{minipage} & \begin{minipage}[t]{0.71\columnwidth}\raggedright\strut
Permite al usuario gestionar la conexión de la aplicación con el servidor.\strut
\end{minipage}\tabularnewline
\begin{minipage}[t]{0.23\columnwidth}\raggedright\strut
\textbf{Precondición}\strut
\end{minipage} & \begin{minipage}[t]{0.71\columnwidth}\raggedright\strut
El usuario debe encontrarse en la pestaña principal (home).\strut
\end{minipage}\tabularnewline
\begin{minipage}[t]{0.23\columnwidth}\raggedright\strut
\textbf{Acciones}\strut
\end{minipage} & \begin{minipage}[t]{0.71\columnwidth}\raggedright\strut
\begin{enumerate}
\def\labelenumi{\arabic{enumi}.}
\tightlist
\item
  El usuario pulsa el botón del menú lateral superior izquierdo.
\item
  El usuario elige la opción ``\textit{Settings}''.
\item
  Se visualizan los campos IP y Puerto para la conexión con el servidor.

\end{enumerate}\strut
\end{minipage}\tabularnewline
\begin{minipage}[t]{0.23\columnwidth}\raggedright\strut
\textbf{Postcondición}\strut
\end{minipage} & \begin{minipage}[t]{0.71\columnwidth}\raggedright\strut
-\strut
\end{minipage}\tabularnewline
\begin{minipage}[t]{0.23\columnwidth}\raggedright\strut
\textbf{Excepciones}\strut
\end{minipage} & \begin{minipage}[t]{0.71\columnwidth}\raggedright\strut
\begin{itemize}
\tightlist
\item
  Error al cargar campo (mensaje).
\end{itemize}\strut
\end{minipage}\tabularnewline
\begin{minipage}[t]{0.23\columnwidth}\raggedright\strut
\textbf{Importancia}\strut
\end{minipage} & \begin{minipage}[t]{0.71\columnwidth}\raggedright\strut
Alta\strut
\end{minipage}\tabularnewline
\bottomrule
\caption{CU-5 Gestión de la conexión aplicación-servidor.}
\end{longtable}

%CASO DE USO 6 ---------------------------------------------------------

\begin{longtable}[h!]{@{}ll@{}}
\toprule
\begin{minipage}[b]{0.23\columnwidth}\raggedright\strut
\textbf{CU-6}\strut
\end{minipage} & \begin{minipage}[b]{0.71\columnwidth}\raggedright\strut
\textbf{Conectar al servidor}\strut
\end{minipage}\tabularnewline
\midrule
\endhead
\begin{minipage}[t]{0.23\columnwidth}\raggedright\strut
\textbf{Versión}\strut
\end{minipage} & \begin{minipage}[t]{0.71\columnwidth}\raggedright\strut
1.0\strut
\end{minipage}\tabularnewline
\begin{minipage}[t]{0.23\columnwidth}\raggedright\strut
\textbf{Autor}\strut
\end{minipage} & \begin{minipage}[t]{0.71\columnwidth}\raggedright\strut
\nombre\strut
\end{minipage}\tabularnewline
\begin{minipage}[t]{0.23\columnwidth}\raggedright\strut
\textbf{Requisitos asociados}\strut
\end{minipage} & \begin{minipage}[t]{0.71\columnwidth}\raggedright\strut
RF-2.1\strut
\end{minipage}\tabularnewline
\begin{minipage}[t]{0.23\columnwidth}\raggedright\strut
\textbf{Descripción}\strut
\end{minipage} & \begin{minipage}[t]{0.71\columnwidth}\raggedright\strut
Permite al usuario conectarse al servidor desde la aplicación.\strut
\end{minipage}\tabularnewline
\begin{minipage}[t]{0.23\columnwidth}\raggedright\strut
\textbf{Precondición}\strut
\end{minipage} & \begin{minipage}[t]{0.71\columnwidth}\raggedright\strut
El usuario debe encontrarse en la pestaña de ajustes (settings).
La aplicación no debe estar conectada al servidor.\strut
\end{minipage}\tabularnewline
\begin{minipage}[t]{0.23\columnwidth}\raggedright\strut
\textbf{Acciones}\strut
\end{minipage} & \begin{minipage}[t]{0.71\columnwidth}\raggedright\strut
\begin{enumerate}
\def\labelenumi{\arabic{enumi}.}
\tightlist
\item
  El usuario rellena los campos ``\textit{IP}'' y ``\textit{Puerto}'' con datos válidos.
\item
  El usuario pulsa el botón ``\textit{Conectarse con el servidor}''.
\item
  Se muestra un mensaje con el estado de la conexión.

\end{enumerate}\strut
\end{minipage}\tabularnewline
\begin{minipage}[t]{0.23\columnwidth}\raggedright\strut
\textbf{Postcondición}\strut
\end{minipage} & \begin{minipage}[t]{0.71\columnwidth}\raggedright\strut
La aplicación mantiene una conexión con el servidor.\strut
\end{minipage}\tabularnewline
\begin{minipage}[t]{0.23\columnwidth}\raggedright\strut
\textbf{Excepciones}\strut
\end{minipage} & \begin{minipage}[t]{0.71\columnwidth}\raggedright\strut
\begin{itemize}
\tightlist
\item
  Error al conectar al servidor (mensaje).
\item
  Destino inalcanzable (mensaje).
\end{itemize}\strut
\end{minipage}\tabularnewline
\begin{minipage}[t]{0.23\columnwidth}\raggedright\strut
\textbf{Importancia}\strut
\end{minipage} & \begin{minipage}[t]{0.71\columnwidth}\raggedright\strut
Alta\strut
\end{minipage}\tabularnewline
\bottomrule
\caption{CU-6 Conectar al servidor.}
\end{longtable}

%CASO DE USO 7 ---------------------------------------------------------

\begin{longtable}[h!]{@{}ll@{}}
\toprule
\begin{minipage}[b]{0.23\columnwidth}\raggedright\strut
\textbf{CU-7}\strut
\end{minipage} & \begin{minipage}[b]{0.71\columnwidth}\raggedright\strut
\textbf{Desconectar del servidor.}\strut
\end{minipage}\tabularnewline
\midrule
\endhead
\begin{minipage}[t]{0.23\columnwidth}\raggedright\strut
\textbf{Versión}\strut
\end{minipage} & \begin{minipage}[t]{0.71\columnwidth}\raggedright\strut
1.0\strut
\end{minipage}\tabularnewline
\begin{minipage}[t]{0.23\columnwidth}\raggedright\strut
\textbf{Autor}\strut
\end{minipage} & \begin{minipage}[t]{0.71\columnwidth}\raggedright\strut
\nombre\strut
\end{minipage}\tabularnewline
\begin{minipage}[t]{0.23\columnwidth}\raggedright\strut
\textbf{Requisitos asociados}\strut
\end{minipage} & \begin{minipage}[t]{0.71\columnwidth}\raggedright\strut
RF-2.2\strut
\end{minipage}\tabularnewline
\begin{minipage}[t]{0.23\columnwidth}\raggedright\strut
\textbf{Descripción}\strut
\end{minipage} & \begin{minipage}[t]{0.71\columnwidth}\raggedright\strut
Permite al usuario desconectarse del servidor desde la aplicación.\strut
\end{minipage}\tabularnewline
\begin{minipage}[t]{0.23\columnwidth}\raggedright\strut
\textbf{Precondición}\strut
\end{minipage} & \begin{minipage}[t]{0.71\columnwidth}\raggedright\strut
La aplicación debe estar conectada al servidor.\strut
\end{minipage}\tabularnewline
\begin{minipage}[t]{0.23\columnwidth}\raggedright\strut
\textbf{Acciones}\strut
\end{minipage} & \begin{minipage}[t]{0.71\columnwidth}\raggedright\strut
\begin{enumerate}
\def\labelenumi{\arabic{enumi}.}
\tightlist
\item
  El usuario pulsa el botón del menú desplegable superior derecho.
\item
  El usuario elige la opción ``\textit{Disconnect}''.
\item
  Se muestra un mensaje indicando que se ha desconectado del servidor.
\end{enumerate}\strut
\end{minipage}\tabularnewline
\begin{minipage}[t]{0.23\columnwidth}\raggedright\strut
\textbf{Postcondición}\strut
\end{minipage} & \begin{minipage}[t]{0.71\columnwidth}\raggedright\strut
La aplicación ha cerrado la conexión con el servidor.\strut
\end{minipage}\tabularnewline
\begin{minipage}[t]{0.23\columnwidth}\raggedright\strut
\textbf{Excepciones}\strut
\end{minipage} & \begin{minipage}[t]{0.71\columnwidth}\raggedright\strut
\begin{itemize}
\tightlist
\item
  Error al desconectar del servidor (mensaje).
\end{itemize}\strut
\end{minipage}\tabularnewline
\begin{minipage}[t]{0.23\columnwidth}\raggedright\strut
\textbf{Importancia}\strut
\end{minipage} & \begin{minipage}[t]{0.71\columnwidth}\raggedright\strut
Alta\strut
\end{minipage}\tabularnewline
\bottomrule
\caption{CU-7 Desconectar del servidor.}
\end{longtable}

%CASO DE USO 8 ---------------------------------------------------------

\begin{longtable}[h!]{@{}ll@{}}
\toprule
\begin{minipage}[b]{0.23\columnwidth}\raggedright\strut
\textbf{CU-8}\strut
\end{minipage} & \begin{minipage}[b]{0.71\columnwidth}\raggedright\strut
\textbf{Monitorización de las cámaras.}\strut
\end{minipage}\tabularnewline
\midrule
\endhead
\begin{minipage}[t]{0.23\columnwidth}\raggedright\strut
\textbf{Versión}\strut
\end{minipage} & \begin{minipage}[t]{0.71\columnwidth}\raggedright\strut
1.0\strut
\end{minipage}\tabularnewline
\begin{minipage}[t]{0.23\columnwidth}\raggedright\strut
\textbf{Autor}\strut
\end{minipage} & \begin{minipage}[t]{0.71\columnwidth}\raggedright\strut
\nombre\strut
\end{minipage}\tabularnewline
\begin{minipage}[t]{0.23\columnwidth}\raggedright\strut
\textbf{Requisitos asociados}\strut
\end{minipage} & \begin{minipage}[t]{0.71\columnwidth}\raggedright\strut
RF-3, RF-3.1, RF-3.2\strut
\end{minipage}\tabularnewline
\begin{minipage}[t]{0.23\columnwidth}\raggedright\strut
\textbf{Descripción}\strut
\end{minipage} & \begin{minipage}[t]{0.71\columnwidth}\raggedright\strut
Permite al usuario monitorizar las cámaras desde la aplicación.\strut
\end{minipage}\tabularnewline
\begin{minipage}[t]{0.23\columnwidth}\raggedright\strut
\textbf{Precondición}\strut
\end{minipage} & \begin{minipage}[t]{0.71\columnwidth}\raggedright\strut
La aplicación debe estar conectada al servidor.\strut
\end{minipage}\tabularnewline
\begin{minipage}[t]{0.23\columnwidth}\raggedright\strut
\textbf{Acciones}\strut
\end{minipage} & \begin{minipage}[t]{0.71\columnwidth}\raggedright\strut
\begin{enumerate}
\def\labelenumi{\arabic{enumi}.}
\tightlist
\item
  El usuario entra en la aplicación.
\item
  Se muestran todos los botones con los nombres de las cámaras disponibles.
\end{enumerate}\strut
\end{minipage}\tabularnewline
\begin{minipage}[t]{0.23\columnwidth}\raggedright\strut
\textbf{Postcondición}\strut
\end{minipage} & \begin{minipage}[t]{0.71\columnwidth}\raggedright\strut
-\strut
\end{minipage}\tabularnewline
\begin{minipage}[t]{0.23\columnwidth}\raggedright\strut
\textbf{Excepciones}\strut
\end{minipage} & \begin{minipage}[t]{0.71\columnwidth}\raggedright\strut
\begin{itemize}
\tightlist
\item
  Error al cargar cámaras (mensaje).
\end{itemize}\strut
\end{minipage}\tabularnewline
\begin{minipage}[t]{0.23\columnwidth}\raggedright\strut
\textbf{Importancia}\strut
\end{minipage} & \begin{minipage}[t]{0.71\columnwidth}\raggedright\strut
Alta\strut
\end{minipage}\tabularnewline
\bottomrule
\caption{CU-8 Monitorización de las cámaras.}
\end{longtable}

%CASO DE USO 9 ---------------------------------------------------------

\begin{longtable}[h!]{@{}ll@{}}
\toprule
\begin{minipage}[b]{0.23\columnwidth}\raggedright\strut
\textbf{CU-9}\strut
\end{minipage} & \begin{minipage}[b]{0.71\columnwidth}\raggedright\strut
\textbf{Visualizar Imagen.}\strut
\end{minipage}\tabularnewline
\midrule
\endhead
\begin{minipage}[t]{0.23\columnwidth}\raggedright\strut
\textbf{Versión}\strut
\end{minipage} & \begin{minipage}[t]{0.71\columnwidth}\raggedright\strut
1.0\strut
\end{minipage}\tabularnewline
\begin{minipage}[t]{0.23\columnwidth}\raggedright\strut
\textbf{Autor}\strut
\end{minipage} & \begin{minipage}[t]{0.71\columnwidth}\raggedright\strut
\nombre\strut
\end{minipage}\tabularnewline
\begin{minipage}[t]{0.23\columnwidth}\raggedright\strut
\textbf{Requisitos asociados}\strut
\end{minipage} & \begin{minipage}[t]{0.71\columnwidth}\raggedright\strut
RF-3.1\strut
\end{minipage}\tabularnewline
\begin{minipage}[t]{0.23\columnwidth}\raggedright\strut
\textbf{Descripción}\strut
\end{minipage} & \begin{minipage}[t]{0.71\columnwidth}\raggedright\strut
Permite al usuario visualizar la imagen de un cámara de las disponibles.\strut
\end{minipage}\tabularnewline
\begin{minipage}[t]{0.23\columnwidth}\raggedright\strut
\textbf{Precondición}\strut
\end{minipage} & \begin{minipage}[t]{0.71\columnwidth}\raggedright\strut
La aplicación debe estar conectada al servidor.\\
El usuario se encuentra en la pantalla principal (Home).\strut
\end{minipage}\tabularnewline
\begin{minipage}[t]{0.23\columnwidth}\raggedright\strut
\textbf{Acciones}\strut
\end{minipage} & \begin{minipage}[t]{0.71\columnwidth}\raggedright\strut
\begin{enumerate}
\def\labelenumi{\arabic{enumi}.}
\tightlist
\item
  El usuario pulsa en una de las cámaras disponibles.
\item
  El servidor recibe la acción de comenzar a enviar las imágenes.
\item
  Se muestra una pantalla con la imagen en vivo de la cámara seleccionada.
\end{enumerate}\strut
\end{minipage}\tabularnewline
\begin{minipage}[t]{0.23\columnwidth}\raggedright\strut
\textbf{Postcondición}\strut
\end{minipage} & \begin{minipage}[t]{0.71\columnwidth}\raggedright\strut
Se ha cambiado de pantalla y se visualiza la imagen.\strut
\end{minipage}\tabularnewline
\begin{minipage}[t]{0.23\columnwidth}\raggedright\strut
\textbf{Excepciones}\strut
\end{minipage} & \begin{minipage}[t]{0.71\columnwidth}\raggedright\strut
\begin{itemize}
\tightlist
\item
  Error al cargar imagen (mensaje).
\item
  Error en la conexión (mensaje).
\end{itemize}\strut
\end{minipage}\tabularnewline
\begin{minipage}[t]{0.23\columnwidth}\raggedright\strut
\textbf{Importancia}\strut
\end{minipage} & \begin{minipage}[t]{0.71\columnwidth}\raggedright\strut
Alta\strut
\end{minipage}\tabularnewline
\bottomrule
\caption{CU-9 Visualizar Imagen.}
\end{longtable}

%CASO DE USO 10 ---------------------------------------------------------

\begin{longtable}[h!]{@{}ll@{}}
\toprule
\begin{minipage}[b]{0.23\columnwidth}\raggedright\strut
\textbf{CU-10}\strut
\end{minipage} & \begin{minipage}[b]{0.71\columnwidth}\raggedright\strut
\textbf{Grabar un vídeo de una cámara.}\strut
\end{minipage}\tabularnewline
\midrule
\endhead
\begin{minipage}[t]{0.23\columnwidth}\raggedright\strut
\textbf{Versión}\strut
\end{minipage} & \begin{minipage}[t]{0.71\columnwidth}\raggedright\strut
1.0\strut
\end{minipage}\tabularnewline
\begin{minipage}[t]{0.23\columnwidth}\raggedright\strut
\textbf{Autor}\strut
\end{minipage} & \begin{minipage}[t]{0.71\columnwidth}\raggedright\strut
\nombre\strut
\end{minipage}\tabularnewline
\begin{minipage}[t]{0.23\columnwidth}\raggedright\strut
\textbf{Requisitos asociados}\strut
\end{minipage} & \begin{minipage}[t]{0.71\columnwidth}\raggedright\strut
RF-3.2\strut
\end{minipage}\tabularnewline
\begin{minipage}[t]{0.23\columnwidth}\raggedright\strut
\textbf{Descripción}\strut
\end{minipage} & \begin{minipage}[t]{0.71\columnwidth}\raggedright\strut
Permite al usuario grabar un pequeño fragmento de vídeo de un cámara de las disponibles.\strut
\end{minipage}\tabularnewline
\begin{minipage}[t]{0.23\columnwidth}\raggedright\strut
\textbf{Precondición}\strut
\end{minipage} & \begin{minipage}[t]{0.71\columnwidth}\raggedright\strut
La aplicación debe estar conectada al servidor.\\
El usuario se encuentra en la pantalla de visualización de una cámara.\strut
\end{minipage}\tabularnewline
\begin{minipage}[t]{0.23\columnwidth}\raggedright\strut
\textbf{Acciones}\strut
\end{minipage} & \begin{minipage}[t]{0.71\columnwidth}\raggedright\strut
\begin{enumerate}
\def\labelenumi{\arabic{enumi}.}
\tightlist
\item
  El usuario visualiza una cámara y pulsa el botón para grabar.
\item
  La aplicación graba un vídeo y lo almacena en el dispositivo.
\end{enumerate}\strut
\end{minipage}\tabularnewline
\begin{minipage}[t]{0.23\columnwidth}\raggedright\strut
\textbf{Postcondición}\strut
\end{minipage} & \begin{minipage}[t]{0.71\columnwidth}\raggedright\strut
Se ha guardado un vídeo con las imágenes vistas por el usuario de la cámara.\strut
\end{minipage}\tabularnewline
\begin{minipage}[t]{0.23\columnwidth}\raggedright\strut
\textbf{Excepciones}\strut
\end{minipage} & \begin{minipage}[t]{0.71\columnwidth}\raggedright\strut
\begin{itemize}
\tightlist
\item
  Error al cargar imagen (mensaje).
\item
  Error en la conexión (mensaje).
\item
  Error en la grabación (mensaje).
\end{itemize}\strut
\end{minipage}\tabularnewline
\begin{minipage}[t]{0.23\columnwidth}\raggedright\strut
\textbf{Importancia}\strut
\end{minipage} & \begin{minipage}[t]{0.71\columnwidth}\raggedright\strut
Alta\strut
\end{minipage}\tabularnewline
\bottomrule
\caption{CU-10 Grabar un vídeo de una cámara.}
\end{longtable}

%CASO DE USO 11 ---------------------------------------------------------

\begin{longtable}[h!]{@{}ll@{}}
\toprule
\begin{minipage}[b]{0.23\columnwidth}\raggedright\strut
\textbf{CU-11}\strut
\end{minipage} & \begin{minipage}[b]{0.71\columnwidth}\raggedright\strut
\textbf{Gestión del servidor.}\strut
\end{minipage}\tabularnewline
\midrule
\endhead
\begin{minipage}[t]{0.23\columnwidth}\raggedright\strut
\textbf{Versión}\strut
\end{minipage} & \begin{minipage}[t]{0.71\columnwidth}\raggedright\strut
1.0\strut
\end{minipage}\tabularnewline
\begin{minipage}[t]{0.23\columnwidth}\raggedright\strut
\textbf{Autor}\strut
\end{minipage} & \begin{minipage}[t]{0.71\columnwidth}\raggedright\strut
\nombre\strut
\end{minipage}\tabularnewline
\begin{minipage}[t]{0.23\columnwidth}\raggedright\strut
\textbf{Requisitos asociados}\strut
\end{minipage} & \begin{minipage}[t]{0.71\columnwidth}\raggedright\strut
RF-4, RF-4.1, RF-4.2\strut
\end{minipage}\tabularnewline
\begin{minipage}[t]{0.23\columnwidth}\raggedright\strut
\textbf{Descripción}\strut
\end{minipage} & \begin{minipage}[t]{0.71\columnwidth}\raggedright\strut
Permite al usuario gestionar el servidor.\strut
\end{minipage}\tabularnewline
\begin{minipage}[t]{0.23\columnwidth}\raggedright\strut
\textbf{Precondición}\strut
\end{minipage} & \begin{minipage}[t]{0.71\columnwidth}\raggedright\strut
-\strut
\end{minipage}\tabularnewline
\begin{minipage}[t]{0.23\columnwidth}\raggedright\strut
\textbf{Acciones}\strut
\end{minipage} & \begin{minipage}[t]{0.71\columnwidth}\raggedright\strut
\begin{enumerate}
\def\labelenumi{\arabic{enumi}.}
\tightlist
\item
  El usuario inicia la aplicación.
\item
  Se muestra la pantalla de inicio.
\end{enumerate}\strut
\end{minipage}\tabularnewline
\begin{minipage}[t]{0.23\columnwidth}\raggedright\strut
\textbf{Postcondición}\strut
\end{minipage} & \begin{minipage}[t]{0.71\columnwidth}\raggedright\strut
-\strut
\end{minipage}\tabularnewline
\begin{minipage}[t]{0.23\columnwidth}\raggedright\strut
\textbf{Excepciones}\strut
\end{minipage} & \begin{minipage}[t]{0.71\columnwidth}\raggedright\strut
-\strut
\end{minipage}\tabularnewline
\begin{minipage}[t]{0.23\columnwidth}\raggedright\strut
\textbf{Importancia}\strut
\end{minipage} & \begin{minipage}[t]{0.71\columnwidth}\raggedright\strut
Alta\strut
\end{minipage}\tabularnewline
\bottomrule
\caption{CU-11 Gestión del servidor.}
\end{longtable}

%CASO DE USO 12 ---------------------------------------------------------

\begin{longtable}[h!]{@{}ll@{}}
\toprule
\begin{minipage}[b]{0.23\columnwidth}\raggedright\strut
\textbf{CU-12}\strut
\end{minipage} & \begin{minipage}[b]{0.71\columnwidth}\raggedright\strut
\textbf{Iniciar servidor.}\strut
\end{minipage}\tabularnewline
\midrule
\endhead
\begin{minipage}[t]{0.23\columnwidth}\raggedright\strut
\textbf{Versión}\strut
\end{minipage} & \begin{minipage}[t]{0.71\columnwidth}\raggedright\strut
1.0\strut
\end{minipage}\tabularnewline
\begin{minipage}[t]{0.23\columnwidth}\raggedright\strut
\textbf{Autor}\strut
\end{minipage} & \begin{minipage}[t]{0.71\columnwidth}\raggedright\strut
\nombre\strut
\end{minipage}\tabularnewline
\begin{minipage}[t]{0.23\columnwidth}\raggedright\strut
\textbf{Requisitos asociados}\strut
\end{minipage} & \begin{minipage}[t]{0.71\columnwidth}\raggedright\strut
RF-4.1\strut
\end{minipage}\tabularnewline
\begin{minipage}[t]{0.23\columnwidth}\raggedright\strut
\textbf{Descripción}\strut
\end{minipage} & \begin{minipage}[t]{0.71\columnwidth}\raggedright\strut
Permite al usuario iniciar el servidor.\strut
\end{minipage}\tabularnewline
\begin{minipage}[t]{0.23\columnwidth}\raggedright\strut
\textbf{Precondición}\strut
\end{minipage} & \begin{minipage}[t]{0.71\columnwidth}\raggedright\strut
Encontrase en la ventana de comandos de una terminal, en la carpeta que contenga los archivos del servidor.\\
Tener instalado Python 3 en el dispositivo.\strut
\end{minipage}\tabularnewline
\begin{minipage}[t]{0.23\columnwidth}\raggedright\strut
\textbf{Acciones}\strut
\end{minipage} & \begin{minipage}[t]{0.71\columnwidth}\raggedright\strut
\begin{enumerate}
\def\labelenumi{\arabic{enumi}.}
\tightlist
\item
  El usuario ejecuta el archivo ``\textit{MainServer}'' con el comando ``\textit{python3 MainServer.py}''.
\item
  El servidor se inicia y comienza a conectarse a las cámaras que tenía en al base de datos.
\item
  El servidor queda en espera de una conexión.
\end{enumerate}\strut
\end{minipage}\tabularnewline
\begin{minipage}[t]{0.23\columnwidth}\raggedright\strut
\textbf{Postcondición}\strut
\end{minipage} & \begin{minipage}[t]{0.71\columnwidth}\raggedright\strut
El servidor está iniciado en escucha.\\
Se ha establecido conexión con las cámaras disponibles\strut
\end{minipage}\tabularnewline
\begin{minipage}[t]{0.23\columnwidth}\raggedright\strut
\textbf{Excepciones}\strut
\end{minipage} & \begin{minipage}[t]{0.71\columnwidth}\raggedright\strut
\begin{itemize}
\tightlist
\item
  Error al iniciar servidor (mensaje).
\item
  Error al establecer conexión con cámara (mensaje).
\end{itemize}\strut
\end{minipage}\tabularnewline
\begin{minipage}[t]{0.23\columnwidth}\raggedright\strut
\textbf{Importancia}\strut
\end{minipage} & \begin{minipage}[t]{0.71\columnwidth}\raggedright\strut
Alta\strut
\end{minipage}\tabularnewline
\bottomrule
\caption{CU-12 Iniciar servidor.}
\end{longtable}

%CASO DE USO 13 ---------------------------------------------------------

\begin{longtable}[h!]{@{}ll@{}}
\toprule
\begin{minipage}[b]{0.23\columnwidth}\raggedright\strut
\textbf{CU-13}\strut
\end{minipage} & \begin{minipage}[b]{0.71\columnwidth}\raggedright\strut
\textbf{Parar servidor.}\strut
\end{minipage}\tabularnewline
\midrule
\endhead
\begin{minipage}[t]{0.23\columnwidth}\raggedright\strut
\textbf{Versión}\strut
\end{minipage} & \begin{minipage}[t]{0.71\columnwidth}\raggedright\strut
1.0\strut
\end{minipage}\tabularnewline
\begin{minipage}[t]{0.23\columnwidth}\raggedright\strut
\textbf{Autor}\strut
\end{minipage} & \begin{minipage}[t]{0.71\columnwidth}\raggedright\strut
\nombre\strut
\end{minipage}\tabularnewline
\begin{minipage}[t]{0.23\columnwidth}\raggedright\strut
\textbf{Requisitos asociados}\strut
\end{minipage} & \begin{minipage}[t]{0.71\columnwidth}\raggedright\strut
RF-4.2\strut
\end{minipage}\tabularnewline
\begin{minipage}[t]{0.23\columnwidth}\raggedright\strut
\textbf{Descripción}\strut
\end{minipage} & \begin{minipage}[t]{0.71\columnwidth}\raggedright\strut
Permite al usuario parar el servidor.\strut
\end{minipage}\tabularnewline
\begin{minipage}[t]{0.23\columnwidth}\raggedright\strut
\textbf{Precondición}\strut
\end{minipage} & \begin{minipage}[t]{0.71\columnwidth}\raggedright\strut
La aplicación debe estar conectada al servidor.\\
El usuario se encuentra en la pantalla principal (Home).\strut
\end{minipage}\tabularnewline
\begin{minipage}[t]{0.23\columnwidth}\raggedright\strut
\textbf{Acciones}\strut
\end{minipage} & \begin{minipage}[t]{0.71\columnwidth}\raggedright\strut
\begin{enumerate}
\def\labelenumi{\arabic{enumi}.}
\tightlist
\item
  El usuario pulsa el botón del menú desplegable superior derecho.
\item
  El usuario elige la opción ``\textit{Stop Server}''.
\item
  Se muestra un mensaje indicando que se ha parado el servidor.
\end{enumerate}\strut
\end{minipage}\tabularnewline
\begin{minipage}[t]{0.23\columnwidth}\raggedright\strut
\textbf{Postcondición}\strut
\end{minipage} & \begin{minipage}[t]{0.71\columnwidth}\raggedright\strut
El servidor se ha parado y se han cerrado las conexiones.\strut
\end{minipage}\tabularnewline
\begin{minipage}[t]{0.23\columnwidth}\raggedright\strut
\textbf{Excepciones}\strut
\end{minipage} & \begin{minipage}[t]{0.71\columnwidth}\raggedright\strut
\begin{itemize}
\tightlist
\item
  Error al para servidor (mensaje).
\item
  Error en la conexión (mensaje).
\end{itemize}\strut
\end{minipage}\tabularnewline
\begin{minipage}[t]{0.23\columnwidth}\raggedright\strut
\textbf{Importancia}\strut
\end{minipage} & \begin{minipage}[t]{0.71\columnwidth}\raggedright\strut
Alta\strut
\end{minipage}\tabularnewline
\bottomrule
\caption{CU-13 Parar servidor.}
\end{longtable}

%CASO DE USO 14 ---------------------------------------------------------

\begin{longtable}[h!]{@{}ll@{}}
\toprule
\begin{minipage}[b]{0.23\columnwidth}\raggedright\strut
\textbf{CU-14}\strut
\end{minipage} & \begin{minipage}[b]{0.71\columnwidth}\raggedright\strut
\textbf{Gestión del log del sistema}\strut
\end{minipage}\tabularnewline
\midrule
\endhead
\begin{minipage}[t]{0.23\columnwidth}\raggedright\strut
\textbf{Versión}\strut
\end{minipage} & \begin{minipage}[t]{0.71\columnwidth}\raggedright\strut
1.0\strut
\end{minipage}\tabularnewline
\begin{minipage}[t]{0.23\columnwidth}\raggedright\strut
\textbf{Autor}\strut
\end{minipage} & \begin{minipage}[t]{0.71\columnwidth}\raggedright\strut
\nombre\strut
\end{minipage}\tabularnewline
\begin{minipage}[t]{0.23\columnwidth}\raggedright\strut
\textbf{Requisitos asociados}\strut
\end{minipage} & \begin{minipage}[t]{0.71\columnwidth}\raggedright\strut
RF-5, RF-5.1\strut
\end{minipage}\tabularnewline
\begin{minipage}[t]{0.23\columnwidth}\raggedright\strut
\textbf{Descripción}\strut
\end{minipage} & \begin{minipage}[t]{0.71\columnwidth}\raggedright\strut
Permite al usuario gestionar el log del sistema.\strut
\end{minipage}\tabularnewline
\begin{minipage}[t]{0.23\columnwidth}\raggedright\strut
\textbf{Precondición}\strut
\end{minipage} & \begin{minipage}[t]{0.71\columnwidth}\raggedright\strut
El usuario debe encontrarse en la pestaña principal (home).\strut
\end{minipage}\tabularnewline
\begin{minipage}[t]{0.23\columnwidth}\raggedright\strut
\textbf{Acciones}\strut
\end{minipage} & \begin{minipage}[t]{0.71\columnwidth}\raggedright\strut
\begin{enumerate}
\def\labelenumi{\arabic{enumi}.}
\tightlist
\item
  El usuario pulsa el botón del menú lateral superior izquierdo.
\item
  El usuario elige la opción ``\textit{Log}''.
\item
  Se visualiza la pantalla \textit{Log}.

\end{enumerate}\strut
\end{minipage}\tabularnewline
\begin{minipage}[t]{0.23\columnwidth}\raggedright\strut
\textbf{Postcondición}\strut
\end{minipage} & \begin{minipage}[t]{0.71\columnwidth}\raggedright\strut
-\strut
\end{minipage}\tabularnewline
\begin{minipage}[t]{0.23\columnwidth}\raggedright\strut
\textbf{Excepciones}\strut
\end{minipage} & \begin{minipage}[t]{0.71\columnwidth}\raggedright\strut
\begin{itemize}
\tightlist
\item
  Error al cargar log (mensaje).
\end{itemize}\strut
\end{minipage}\tabularnewline
\begin{minipage}[t]{0.23\columnwidth}\raggedright\strut
\textbf{Importancia}\strut
\end{minipage} & \begin{minipage}[t]{0.71\columnwidth}\raggedright\strut
Alta\strut
\end{minipage}\tabularnewline
\bottomrule
\caption{CU-14 Gestión del log del sistema.}
\end{longtable}

%CASO DE USO 15 ---------------------------------------------------------

\begin{longtable}[h!]{@{}ll@{}}
\toprule
\begin{minipage}[b]{0.23\columnwidth}\raggedright\strut
\textbf{CU-15}\strut
\end{minipage} & \begin{minipage}[b]{0.71\columnwidth}\raggedright\strut
\textbf{Visualizar log del sistema}\strut
\end{minipage}\tabularnewline
\midrule
\endhead
\begin{minipage}[t]{0.23\columnwidth}\raggedright\strut
\textbf{Versión}\strut
\end{minipage} & \begin{minipage}[t]{0.71\columnwidth}\raggedright\strut
1.0\strut
\end{minipage}\tabularnewline
\begin{minipage}[t]{0.23\columnwidth}\raggedright\strut
\textbf{Autor}\strut
\end{minipage} & \begin{minipage}[t]{0.71\columnwidth}\raggedright\strut
\nombre\strut
\end{minipage}\tabularnewline
\begin{minipage}[t]{0.23\columnwidth}\raggedright\strut
\textbf{Requisitos asociados}\strut
\end{minipage} & \begin{minipage}[t]{0.71\columnwidth}\raggedright\strut
RF-5.1\strut
\end{minipage}\tabularnewline
\begin{minipage}[t]{0.23\columnwidth}\raggedright\strut
\textbf{Descripción}\strut
\end{minipage} & \begin{minipage}[t]{0.71\columnwidth}\raggedright\strut
Permite al usuario visualizar el log del sistema.\strut
\end{minipage}\tabularnewline
\begin{minipage}[t]{0.23\columnwidth}\raggedright\strut
\textbf{Precondición}\strut
\end{minipage} & \begin{minipage}[t]{0.71\columnwidth}\raggedright\strut
El usuario debe encontrarse en la pestaña del log (Log).\strut
\end{minipage}\tabularnewline
\begin{minipage}[t]{0.23\columnwidth}\raggedright\strut
\textbf{Acciones}\strut
\end{minipage} & \begin{minipage}[t]{0.71\columnwidth}\raggedright\strut
\begin{enumerate}
\def\labelenumi{\arabic{enumi}.}
\tightlist
\item
  Se muestra el log del sistema en formato texto.
\item
  El usuario puede hacer \textit{scroll} para ver el contenido del log desde el incio de la aplicación.
\end{enumerate}\strut
\end{minipage}\tabularnewline
\begin{minipage}[t]{0.23\columnwidth}\raggedright\strut
\textbf{Postcondición}\strut
\end{minipage} & \begin{minipage}[t]{0.71\columnwidth}\raggedright\strut
-\strut
\end{minipage}\tabularnewline
\begin{minipage}[t]{0.23\columnwidth}\raggedright\strut
\textbf{Excepciones}\strut
\end{minipage} & \begin{minipage}[t]{0.71\columnwidth}\raggedright\strut
\begin{itemize}
\tightlist
\item
  Error al cargar log (mensaje).
\end{itemize}\strut
\end{minipage}\tabularnewline
\begin{minipage}[t]{0.23\columnwidth}\raggedright\strut
\textbf{Importancia}\strut
\end{minipage} & \begin{minipage}[t]{0.71\columnwidth}\raggedright\strut
Alta\strut
\end{minipage}\tabularnewline
\bottomrule
\caption{CU-15 Visualizar log del sistema.}
\end{longtable}