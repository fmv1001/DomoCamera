\capitulo{7}{Conclusiones y Líneas de trabajo futuras}

\section{Conclusiones}

Procedo a mostrar las conclusiones que se derivan del desarrollo del presente proyecto:

\begin{itemize}
\tightlist
\item
  El objetivo general del proyecto se ha cumplido con éxito. Se ha desarrollado un entorno en el que se pueden monitorizar cámaras dispuestas en un hogar, manteniendo una interfaz de usuario sencilla y con una corta curva de aprendizaje, llegando a un gran número de dispositivos (99,8\% de los dispositivos Android).
\item
  Se ha conseguido desplegar el servidor sobre un dispositivo de bajo consumo (Raspberry Pi).
\item
  Gracias a la realización del proyecto he profundizado en el conocimiento sobre conexiones con sockets, aplicándolo a un entorno real.
\item
  Este proyecto ha abarcado una buena parte de los conocimientos que he obtenido durante mi estancia en la universidad. Pero no sólo he hecho uso de ellos, si no que también he estudiado y adquirido nuevos conocimientos requeridos para la realización del proyecto. Entre ellos se encuentran: Android, OpenCV, \LaTeX , SonarQube o RTSP.
\item
  Manejar la plataforma SonarQube para el análisis del código me ha ayudado a la detección temprana fallos, defectos y vulnerabilidades en el software, generando un código de mayor calidad y evitando que exista un comportamiento indefinido que pueda afectar a los usuarios finales.
\item
  Gracias a la investigación que ha requerido la realización del proyecto, se ha aprendido
a realizar rápidas y eficientes búsquedas bibliográficas.
\item
  Durante el periodo de desarrollo de este proyecto se han manejado un gran numero de tecnologías y herramientas. Todas ellas han ayudado a mejorar la calidad del mismo.
No obstante, algunas de ellas han conllevado una sobrecarga de trabajo importante. 
Pese a ello, el conocimiento adquirido será de gran utilidad en el futuros en otros proyectos.

\end{itemize}



\section{Líneas de trabajo futuras}

A continuación vamos a comentar una a una las posibles mejoras o continuaciones aplicables al presente trabajo.

\subsection{Servidor remoto}

Podría desarrollarse un servidor en la nube para poder acceder a las cámaras desde cualquier parte del mundo con sólo una conexión a Internet.

\subsection{Vista previa}

Podría implementarse sobre la aplicación para que antes de entrar en la cámara se vea en pequeño la imagen.

\subsection{Detección de movimiento}

Detección de movimiento cuando algo se mueva en el rango que se este visualizando.

\subsection{Notificaciones}

Notificaciones en la aplicación.

\subsection{Añadir otros dispositivos del hogar para controlar y/o monitorizar}

Dado que este proyecto tiene un pequeño enfoque domótico, se podrían añadir fácilmente otros elementos para ser controlados y monitorizados, ya sea la calefacción (poder programarlas, encenderla o apagarla desde el teléfono), la iluminación, una lavadora, las persianas (programar cuando subirlas y bajarlas para que el sol caliente la casa por la mañana en invierno, pero no entre demasiado el frío por la noche), u otros dispositivos inteligentes del hogar.

\subsection{Implementar motorización desde PC}

Se podría implementar una herramienta en Windows o Linux para monitorizar el sistema, a la vez que en android.









