\capitulo{5}{Aspectos relevantes del desarrollo del proyecto}

Este apartado pretende recoger los aspectos más interesantes del desarrollo del proyecto, desde la exposición del ciclo de vida utilizado, hasta los detalles de mayor relevancia de las fases de análisis, diseño e implementación.

\section{Elección del tema}

Siempre me ha gustado el hecho de desarrollar una aplicación Android y cuando estuve mirando tema para mi proyecto, me llamó la atención el hecho de desarrollar algo que me sirviera en mi día a día, se ofertaba un trabajo para hacer una aplicación con Unity3D, pero el profesor había cambiado de área y ya no lo tutorizaba. 
Así que decidí darle mi enfoque y realizar la aplicación para el sistema Android dado lo extendido que está. 
También decidí hacer el servidor dado que una asignatura que he cursado en este último curso trataba el tema y me llamó mucho la atención. Y todo esto me llevó a realizar el proyecto que expongo a continuación.



\section{Comienzo del proyecto}

Una vez escogido el tema, tocaba diseñar la estructura del proyecto. 
Decidí que la mejor opción sería separar el servidor en un dispositivo y la aplicación en otro, ya que así el dispositivo no cargaría con toda la carga computacional que supondría para él, dado que estos no disponen de mucha batería, aplicando el modelo cliente-servidor.\\
Ahora tocaba decidir donde desarrollar el servidor y en que lenguaje. En este paso, me decanté por el lenguaje Python ya que está muy extendido (llegando a ser uno de los lenguajes de programación mas usado del mundo \cite{pythonuse}) y estaba muy familiarizado con él debido a que lo he usado en numerosas ocasiones.\\
Después tenía que asegurarme de poder realizar la conexión con una cámara a través de Python.
Tras una búsqueda intensa por internet encontré la librería OpenCV. 
Para la realización de este proyecto me compré una cámara IP, pero tras unos días no conseguía acceder a ella desde OpenCV con python. 
Para no perder demasiado tiempo contacté con la empresa que me había vendido la cámara y les comenté mi problema.
Rápidamente me contestaron y me proporcionaron la dirección que debía usar, y gracias a ello pude realizar una conexión servidor-cámara.\\
Una vez realizada dicha conexión ya podía empezar con el siguiente punto del proyecto.



\section{Desarrollo de la aplicación y servidor}


























